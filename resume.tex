%%%%%%%%%%%%%%%%%%%%%%%%%%%%%%%%%%%%%%%
% Deedy - One Page Two Column Resume
% LaTeX Template
% Version 1.2 (16/9/2014)
%
% Original author:
% Debarghya Das (http://debarghyadas.com)
%
% Original repository:
% https://github.com/deedydas/Deedy-Resume
%
% IMPORTANT: THIS TEMPLATE NEEDS TO BE COMPILED WITH XeLaTeX
%
% This template uses several fonts not included with Windows/Linux by
% default. If you get compilation errors saying a font is missing, find the line
% on which the font is used and either change it to a font included with your
% operating system or comment the line out to use the default font.
% 
%%%%%%%%%%%%%%%%%%%%%%%%%%%%%%%%%%%%%%
% 
% TODO:
% 1. Integrate biber/bibtex for article citation under publications.
% 2. Figure out a smoother way for the document to flow onto the next page.
% 3. Add styling information for a "Projects/Hacks" section.
% 4. Add location/address information
% 5. Merge OpenFont and MacFonts as a single sty with options.
% 
%%%%%%%%%%%%%%%%%%%%%%%%%%%%%%%%%%%%%%
%
% CHANGELOG:
% v1.1:
% 1. Fixed several compilation bugs with \renewcommand
% 2. Got Open-source fonts (Windows/Linux support)
% 3. Added Last Updated
% 4. Move Title styling into .sty
% 5. Commented .sty file.
%
%%%%%%%%%%%%%%%%%%%%%%%%%%%%%%%%%%%%%%%
%
% Known Issues:
% 1. Overflows onto second page if any column's contents are more than the
% vertical limit
% 2. Hacky space on the first bullet point on the second column.
%
%%%%%%%%%%%%%%%%%%%%%%%%%%%%%%%%%%%%%%


\documentclass[]{deedy-resume-openfont}
\usepackage{fancyhdr}
 
\pagestyle{fancy}
\fancyhf{}
 
\begin{document}

%%%%%%%%%%%%%%%%%%%%%%%%%%%%%%%%%%%%%%
%
%     LAST UPDATED DATE
%
%%%%%%%%%%%%%%%%%%%%%%%%%%%%%%%%%%%%%%
\lastupdated

%%%%%%%%%%%%%%%%%%%%%%%%%%%%%%%%%%%%%%
%
%     TITLE NAME
%
%%%%%%%%%%%%%%%%%%%%%%%%%%%%%%%%%%%%%%
\namesection{Daniel}{Weber}{ \urlstyle{same}
\href{mailto:dweber11@jhu.edu}{\underline{dweber11@jhu.edu}} | 917.200.7111 | \href{https://www.linkedin.com/in/daniel-k-weber/}{\underline{linkedin/daniel-k-weber}} | \href{https://github.com/Danielkweber}{\underline{github/danielkweber}} 
}

%%%%%%%%%%%%%%%%%%%%%%%%%%%%%%%%%%%%%%
%     EDUCATION
%%%%%%%%%%%%%%%%%%%%%%%%%%%%%%%%%%%%%%
\section{Education} 
\runsubsection{Johns Hopkins University}
\location{May 2023 | Baltimore, Maryland }

\descript{BS Computer Science | BS Applied Mathematics and Statistics | BA Mathematics}

\location{\quad | GPA: 3.98 / 4.0 \quad | Major GPA: 4.0 / 4.0 \quad |  Dean's List (2019-2023) }

%%%%%%%%%%%%%%%%%%%%%%%%%%%%%%%%%%%%%%
%     EXPERIENCE
%%%%%%%%%%%%%%%%%%%%%%%%%%%%%%%%%%%%%%

\section{Experience}
\runsubsection{Handshake}
\descript{| Software Engineer Intern } 
\qquad \quad \location{June 2022 – Aug. 2022 | San Francisco, California (Remote)}
\begin{tightemize}
    \item Proposed an initiative to scale event publishing infrastructure 
    via sharding to achieve an over 10X increase in throughput.
    \item Autoscaled Kubernetes deployments using Datadog metrics to reliably 
    ensure event processing within minutes of publishing.
    \item Proposed and implemented a Google Pub/Sub publishing interface that allows 
    developers to asynchronously publish events while maintaining ordering and data 
    consistency.
    \item Collaborated with developers both on and off-team to push cross-functional goals 
    which improved organizational efficiency and the user-facing experience.
    \item Increased the observability of our publishing systems by designing and integrating effective traces/metrics  
    which allowed myself and peer developers to identify problems thus sparking new engineering initiatives.
    \item Technologies: Ruby, Rails, GCP, Google Pub/Sub, Kubernetes, Helm, Datadog, Github
    \end{tightemize}
\sectionsep

\runsubsection{Amazon.com}
\descript{| Software Engineer Intern }
\qquad\qquad\qquad\qquad\quad \location{May 2021 – Aug. 2021 | Seattle, Washington}
\begin{tightemize}
    \item Designed and implemented a heap dump analysis tool to aid developers in the
    optimization of processes utilizing over 250GB of heap memory.
    \item Collaborated with key stakeholders on the ad infastructure team
    to find the pain points in the existing optimization process.
    \item Implemented automation to give developers easy access to up-to-date 
    heap dump files, shortening an 8 hour process to a minutes long task.
    \item Leveraged CI/CD technology to develop an extensible
    and resilient platform that can evolve with changing business needs.
    \item Technologies: Java, AWS, CodePipelines, Git
    \end{tightemize}
\sectionsep

\runsubsection{Distributed Systems/Networks Lab} \href{http://www.dsn.jhu.edu/courses/cs310/power-grid/}{\faExternalLink}
\descript{| Security Researcher}
\location{Jan. 2021 – May 2021 | Johns Hopkins}
\begin{tightemize}
    \item Curated a deep understanding of a large, intricate code base thus allowing me
    to demonstate key security vulnerabilities at both a protocol and implementation level.
    \item Crafted a resource consumption attack which downed SPIRE, a fault-tolerant 
    distributed system designed to securely control the US power grid, in under 20 minutes.
    \item Presented the discovered vulnerabilities and attacks to the Department of Defense
    who have used it to futher enhance SPIRE.
\end{tightemize}
\sectionsep

\runsubsection{Algorithms/CS Fundamentals} 
\descript{| Teacher's Assistant}
\qquad \quad \location{Sept. 2020 – October 2021 | Johns Hopkins}
\begin{tightemize}
    \item Taught students key algorithmic concepts like complexity analysis, 
    dynamic programming, and graph traversal while holding office hours and grading HW.
    \item Educated students in low-level computing concepts like data representation, 
    memory safety, and parallelism while performing code reviews.
\end{tightemize}
\sectionsep

\runsubsection{Faye}
\descript{| Machine Learning Intern }
\qquad\qquad\qquad\qquad\qquad\qquad\qquad\qquad\quad\location{May 2020 - Sept. 2020 | Tel Aviv, Israel}
\begin{tightemize}
    \item Created a transformer-powered NLU chatbot to replace an 
    off-the-shelf rule-based Google Dialogflow model that dramatically 
    improved customer workflows through context-aware responses and actions.  
    \item Technologies: Python, Javascript, Tensorflow, Rasa, Docker, Google Cloud
    \end{tightemize}
\sectionsep

\runsubsection{David Energy}
\descript{| Machine Learning Engineer }
\qquad\qquad\qquad\qquad\location{July 2019 – Sept. 2019 | Brooklyn, New York}
\begin{tightemize}
\item Curated an extensive dataset of high-quality electricity usage predictors from
both internal and external sources.
\item Developed machine learning models to predict a building's electricity demand with 97\% 
accuracy and deployed said models to allow for real-time electricity usage prediction.
\item Architected a secure AWS cloud solution to allow the company's infastructure to scale 
as more customers joined the platform.
\item Technologies: Python, Git, Tensorflow, Scikit-Learn, AWS
\end{tightemize}

%%%%%%%%%%%%%%%%%%%%%%%%%%%%%%%%%%%%%%
%     SKILLS
%%%%%%%%%%%%%%%%%%%%%%%%%%%%%%%%%%%%%%

\section{Skills}
\vspace{-5pt}
\location{Languages:} 
\cvtag{Python}    
\cvtag{Java} 
\cvtag{C}  
\cvtag{C++}
\cvtag{Ruby}  
\cvtag{Go} 
\cvtag{Javascript} 
\cvtag{Matlab} 
\cvtag{SQL} 
\cvtag{\LaTeX}  
\cvtag{x86 Assembly}

\location{Technologies:}
\cvtag{Git} 
\cvtag{AWS-Certified}
\cvtag{Google Cloud}
\cvtag{Kubernetes} 
\cvtag{Docker}
\cvtag{OOP} 
\cvtag{Rails}
\cvtag{Postgre}
\cvtag{TensorFlow}

\location{Hobbys/Interests:}
\cvtag{Cycling} 
\cvtag{Hiking}  
\cvtag{Coffee Drinking}
\cvtag{Concertgoing}
\cvtag{Piano Playing}
\end{document}  \documentclass[]{article}
